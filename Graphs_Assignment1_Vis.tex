% Options for packages loaded elsewhere
\PassOptionsToPackage{unicode}{hyperref}
\PassOptionsToPackage{hyphens}{url}
%
\documentclass[
]{article}
\usepackage{lmodern}
\usepackage{amssymb,amsmath}
\usepackage{ifxetex,ifluatex}
\ifnum 0\ifxetex 1\fi\ifluatex 1\fi=0 % if pdftex
  \usepackage[T1]{fontenc}
  \usepackage[utf8]{inputenc}
  \usepackage{textcomp} % provide euro and other symbols
\else % if luatex or xetex
  \usepackage{unicode-math}
  \defaultfontfeatures{Scale=MatchLowercase}
  \defaultfontfeatures[\rmfamily]{Ligatures=TeX,Scale=1}
\fi
% Use upquote if available, for straight quotes in verbatim environments
\IfFileExists{upquote.sty}{\usepackage{upquote}}{}
\IfFileExists{microtype.sty}{% use microtype if available
  \usepackage[]{microtype}
  \UseMicrotypeSet[protrusion]{basicmath} % disable protrusion for tt fonts
}{}
\makeatletter
\@ifundefined{KOMAClassName}{% if non-KOMA class
  \IfFileExists{parskip.sty}{%
    \usepackage{parskip}
  }{% else
    \setlength{\parindent}{0pt}
    \setlength{\parskip}{6pt plus 2pt minus 1pt}}
}{% if KOMA class
  \KOMAoptions{parskip=half}}
\makeatother
\usepackage{xcolor}
\IfFileExists{xurl.sty}{\usepackage{xurl}}{} % add URL line breaks if available
\IfFileExists{bookmark.sty}{\usepackage{bookmark}}{\usepackage{hyperref}}
\hypersetup{
  pdftitle={Assigment 1\_VIS},
  pdfauthor={Sage Grey},
  hidelinks,
  pdfcreator={LaTeX via pandoc}}
\urlstyle{same} % disable monospaced font for URLs
\usepackage[margin=1in]{geometry}
\usepackage{color}
\usepackage{fancyvrb}
\newcommand{\VerbBar}{|}
\newcommand{\VERB}{\Verb[commandchars=\\\{\}]}
\DefineVerbatimEnvironment{Highlighting}{Verbatim}{commandchars=\\\{\}}
% Add ',fontsize=\small' for more characters per line
\usepackage{framed}
\definecolor{shadecolor}{RGB}{248,248,248}
\newenvironment{Shaded}{\begin{snugshade}}{\end{snugshade}}
\newcommand{\AlertTok}[1]{\textcolor[rgb]{0.94,0.16,0.16}{#1}}
\newcommand{\AnnotationTok}[1]{\textcolor[rgb]{0.56,0.35,0.01}{\textbf{\textit{#1}}}}
\newcommand{\AttributeTok}[1]{\textcolor[rgb]{0.77,0.63,0.00}{#1}}
\newcommand{\BaseNTok}[1]{\textcolor[rgb]{0.00,0.00,0.81}{#1}}
\newcommand{\BuiltInTok}[1]{#1}
\newcommand{\CharTok}[1]{\textcolor[rgb]{0.31,0.60,0.02}{#1}}
\newcommand{\CommentTok}[1]{\textcolor[rgb]{0.56,0.35,0.01}{\textit{#1}}}
\newcommand{\CommentVarTok}[1]{\textcolor[rgb]{0.56,0.35,0.01}{\textbf{\textit{#1}}}}
\newcommand{\ConstantTok}[1]{\textcolor[rgb]{0.00,0.00,0.00}{#1}}
\newcommand{\ControlFlowTok}[1]{\textcolor[rgb]{0.13,0.29,0.53}{\textbf{#1}}}
\newcommand{\DataTypeTok}[1]{\textcolor[rgb]{0.13,0.29,0.53}{#1}}
\newcommand{\DecValTok}[1]{\textcolor[rgb]{0.00,0.00,0.81}{#1}}
\newcommand{\DocumentationTok}[1]{\textcolor[rgb]{0.56,0.35,0.01}{\textbf{\textit{#1}}}}
\newcommand{\ErrorTok}[1]{\textcolor[rgb]{0.64,0.00,0.00}{\textbf{#1}}}
\newcommand{\ExtensionTok}[1]{#1}
\newcommand{\FloatTok}[1]{\textcolor[rgb]{0.00,0.00,0.81}{#1}}
\newcommand{\FunctionTok}[1]{\textcolor[rgb]{0.00,0.00,0.00}{#1}}
\newcommand{\ImportTok}[1]{#1}
\newcommand{\InformationTok}[1]{\textcolor[rgb]{0.56,0.35,0.01}{\textbf{\textit{#1}}}}
\newcommand{\KeywordTok}[1]{\textcolor[rgb]{0.13,0.29,0.53}{\textbf{#1}}}
\newcommand{\NormalTok}[1]{#1}
\newcommand{\OperatorTok}[1]{\textcolor[rgb]{0.81,0.36,0.00}{\textbf{#1}}}
\newcommand{\OtherTok}[1]{\textcolor[rgb]{0.56,0.35,0.01}{#1}}
\newcommand{\PreprocessorTok}[1]{\textcolor[rgb]{0.56,0.35,0.01}{\textit{#1}}}
\newcommand{\RegionMarkerTok}[1]{#1}
\newcommand{\SpecialCharTok}[1]{\textcolor[rgb]{0.00,0.00,0.00}{#1}}
\newcommand{\SpecialStringTok}[1]{\textcolor[rgb]{0.31,0.60,0.02}{#1}}
\newcommand{\StringTok}[1]{\textcolor[rgb]{0.31,0.60,0.02}{#1}}
\newcommand{\VariableTok}[1]{\textcolor[rgb]{0.00,0.00,0.00}{#1}}
\newcommand{\VerbatimStringTok}[1]{\textcolor[rgb]{0.31,0.60,0.02}{#1}}
\newcommand{\WarningTok}[1]{\textcolor[rgb]{0.56,0.35,0.01}{\textbf{\textit{#1}}}}
\usepackage{graphicx,grffile}
\makeatletter
\def\maxwidth{\ifdim\Gin@nat@width>\linewidth\linewidth\else\Gin@nat@width\fi}
\def\maxheight{\ifdim\Gin@nat@height>\textheight\textheight\else\Gin@nat@height\fi}
\makeatother
% Scale images if necessary, so that they will not overflow the page
% margins by default, and it is still possible to overwrite the defaults
% using explicit options in \includegraphics[width, height, ...]{}
\setkeys{Gin}{width=\maxwidth,height=\maxheight,keepaspectratio}
% Set default figure placement to htbp
\makeatletter
\def\fps@figure{htbp}
\makeatother
\setlength{\emergencystretch}{3em} % prevent overfull lines
\providecommand{\tightlist}{%
  \setlength{\itemsep}{0pt}\setlength{\parskip}{0pt}}
\setcounter{secnumdepth}{-\maxdimen} % remove section numbering

\title{Assigment 1\_VIS}
\author{Sage Grey}
\date{9/12/2020}

\begin{document}
\maketitle

\textbf{Loading Libraries}

\begin{Shaded}
\begin{Highlighting}[]
\KeywordTok{library}\NormalTok{(ggplot2)}
\KeywordTok{library}\NormalTok{(tidycensus)}
\KeywordTok{library}\NormalTok{(tidyverse)}
\KeywordTok{library}\NormalTok{(tinytex)}


\KeywordTok{library}\NormalTok{(wesanderson)}
\KeywordTok{library}\NormalTok{(ggthemes)}
\end{Highlighting}
\end{Shaded}

\hypertarget{overview}{%
\subsection{Overview}\label{overview}}

Hello! Below are 10 graphs I generated using American Community Survey
Data from 2018 for a Neighborhood (Single PUMA) in Albuquerque, NM.

\textbf{Variables Used} 1. Household Income \textbar\textbar{}
\textbf{HINCP} \textbar\textbar{} Categorical 2. When Structure was
First Built \textbar\textbar{} \textbf{YBL} \textbar\textbar{}
Categorical 3. Majority Race \textbar\textbar{} \textbf{RAC1P}
\textbar\textbar{} Categorical 4. First Mortgage Payment
\textbar\textbar{} \textbf{MRGP} \textbar\textbar{} Continuous 5.
Monthly Rent\textbar\textbar{} \textbf{RNTP} \textbar\textbar{}
Continuous 6. Travel Time To Work \textbar\textbar{} \textbf{JWMNP}
\textbar\textbar{} Continuous 7. Location (PUMA) \textbar\textbar{}
\textbf{PUMA}\textbar\textbar{} Categorical 8. Number of Persons in
Household \textbar\textbar{} \textbf{NP}\textbar\textbar{} Continuous 9.
Property Value \textbar\textbar{} \textbf{VALP} \textbar\textbar{}
Continuous 10. Age Person \textbar\textbar{} \textbf{AGEP}
\textbar\textbar{} Continuous

\textbf{Variables Created} 1. Development \textbar\textbar{}
\textbf{development}\textbar\textbar{} Categorical 2. Old Structures
\textbar\textbar{} \textbf{struct\_old}\textbar\textbar{} Categorical 3.
Structures Built in last twenty years \textbar\textbar{}
\textbf{struct\_last\_twenty} \textbar\textbar{} Categorical

\hypertarget{choosing-variables-creating-dataset}{%
\subsection{Choosing Variables \& Creating
DataSet}\label{choosing-variables-creating-dataset}}

\begin{Shaded}
\begin{Highlighting}[]
\NormalTok{ass1_vis_vars <-}\StringTok{  }\KeywordTok{get_pums}\NormalTok{(}\DataTypeTok{variables =} \KeywordTok{c}\NormalTok{(}\StringTok{"YBL"}\NormalTok{, }
                                          \StringTok{"HINCP"}\NormalTok{,}
                                          \StringTok{"RAC1P"}\NormalTok{, }
                                          \StringTok{"JWMNP"}\NormalTok{,}
                                          \StringTok{"MRGP"}\NormalTok{,}
                                          \StringTok{"RNTP"}\NormalTok{,}
                                          \StringTok{"NP"}\NormalTok{, }
                                          \StringTok{"PUMA"}\NormalTok{, }
                                         \StringTok{"VALP"}\NormalTok{, }
                                         \StringTok{"AGEP"}\NormalTok{,}
                                         \StringTok{"HISP"}\NormalTok{),}
                \DataTypeTok{state=} \StringTok{"NM"}\NormalTok{,}
                \DataTypeTok{year =}\DecValTok{2018}\NormalTok{,}
                \DataTypeTok{survey =}\StringTok{"acs1"}\NormalTok{,}
                \DataTypeTok{recode=}\OtherTok{TRUE}\NormalTok{) }\OperatorTok
\StringTok{  }
\StringTok{                }\CommentTok{#Albuquerque PUMAS Only}
\StringTok{                }\KeywordTok{mutate}\NormalTok{(}\DataTypeTok{PUMA =} \KeywordTok{as.numeric}\NormalTok{(PUMA))}\OperatorTok
\StringTok{               }\KeywordTok{filter}\NormalTok{(PUMA }\OperatorTok{==}\StringTok{ }\DecValTok{00803}\NormalTok{,VALP }\OperatorTok{<}\StringTok{ }\DecValTok{2000000}\NormalTok{)}\OperatorTok
\StringTok{                }\KeywordTok{filter}\NormalTok{(YBL_label }\OperatorTok{!=}\StringTok{ "1939 or earlier"}\NormalTok{)}\OperatorTok
\StringTok{  }
\KeywordTok{mutate}\NormalTok{(}\DataTypeTok{development =} \KeywordTok{case_when}\NormalTok{(YBL_label }\OperatorTok{==}\StringTok{ "1939 or earlier"} \OperatorTok{~}\StringTok{ "pre 1940"}\NormalTok{, }
\NormalTok{                        YBL_label }\OperatorTok{==}\StringTok{ "1940 to 1949"} \OperatorTok{~}\StringTok{ "1940s"}\NormalTok{,}
\NormalTok{                        YBL_label }\OperatorTok{==}\StringTok{ "1950 to 1959"} \OperatorTok{~}\StringTok{ "1950s"}\NormalTok{,}
\NormalTok{                        YBL_label }\OperatorTok{==}\StringTok{ "1960 to 1969"} \OperatorTok{~}\StringTok{ "1960s"}\NormalTok{,}
\NormalTok{                        YBL_label }\OperatorTok{==}\StringTok{ "1970 to 1979"} \OperatorTok{~}\StringTok{ "1970s"}\NormalTok{,}
\NormalTok{                        YBL_label }\OperatorTok{==}\StringTok{ "1980 to 1989"} \OperatorTok{~}\StringTok{ "1980s"}\NormalTok{,}
\NormalTok{                        YBL_label }\OperatorTok{==}\StringTok{ "1990 to 1999"} \OperatorTok{~}\StringTok{ "1990s"}\NormalTok{,}
\NormalTok{                        YBL_label }\OperatorTok{==}\StringTok{ "2000 to 2004"} \OperatorTok{~}\StringTok{ "2000s"}\NormalTok{,}
\NormalTok{                        YBL_label }\OperatorTok{==}\StringTok{ "2005"}\OperatorTok{~}\StringTok{ "2000s"}\NormalTok{ ,}
\NormalTok{                        YBL_label }\OperatorTok{==}\StringTok{ "2006"}\OperatorTok{~}\StringTok{ "2000s"}\NormalTok{ ,}
\NormalTok{                        YBL_label }\OperatorTok{==}\StringTok{ "2007"}\OperatorTok{~}\StringTok{ "2000s"}\NormalTok{ ,}
\NormalTok{                        YBL_label }\OperatorTok{==}\StringTok{ "2008"}\OperatorTok{~}\StringTok{ "2000s"}\NormalTok{ ,}
\NormalTok{                        YBL_label }\OperatorTok{==}\StringTok{ "2009"}\OperatorTok{~}\StringTok{ "2000s"}\NormalTok{ ,}
\NormalTok{                               YBL_label }\OperatorTok{==}\StringTok{ "2010"} \OperatorTok{~}\StringTok{ "2010s"}\NormalTok{,}
\NormalTok{                               YBL_label }\OperatorTok{==}\StringTok{ "2011"} \OperatorTok{~}\StringTok{ "2010s"}\NormalTok{,}
\NormalTok{                               YBL_label }\OperatorTok{==}\StringTok{ "2012"} \OperatorTok{~}\StringTok{ "2010s"}\NormalTok{,}
\NormalTok{                               YBL_label }\OperatorTok{==}\StringTok{ "2013"} \OperatorTok{~}\StringTok{ "2010s"}\NormalTok{,}
\NormalTok{                               YBL_label }\OperatorTok{==}\StringTok{ "2014"} \OperatorTok{~}\StringTok{ "2010s"}\NormalTok{,}
\NormalTok{                               YBL_label }\OperatorTok{==}\StringTok{ "2015"} \OperatorTok{~}\StringTok{ "2010s"}\NormalTok{,}
\NormalTok{                               YBL_label }\OperatorTok{==}\StringTok{ "2016"} \OperatorTok{~}\StringTok{ "2010s"}\NormalTok{,}
\NormalTok{                               YBL_label }\OperatorTok{==}\StringTok{ "2017"} \OperatorTok{~}\StringTok{ "2010s"}\NormalTok{,}
\NormalTok{                               YBL_label }\OperatorTok{==}\StringTok{ "2018"} \OperatorTok{~}\StringTok{ "2010s"}\NormalTok{)) }\OperatorTok
\StringTok{    }
\KeywordTok{mutate}\NormalTok{(}\DataTypeTok{struct_old =}\NormalTok{ (YBL_label }\OperatorTok{==}\StringTok{ "1939 or earlier"} \OperatorTok{|}
\StringTok{                       }\NormalTok{YBL_label }\OperatorTok{==}\StringTok{ "1940 to 1949"} \OperatorTok{|}\StringTok{ }
\StringTok{                       }\NormalTok{YBL_label }\OperatorTok{==}\StringTok{ "1950 to 1959"} \OperatorTok{|}\StringTok{ }
\StringTok{                       }\NormalTok{YBL_label }\OperatorTok{==}\StringTok{ "1960 to 1969"}\NormalTok{ )) }\OperatorTok
\StringTok{  }
\KeywordTok{mutate}\NormalTok{(}\DataTypeTok{struct_last_twenty =}\NormalTok{ (YBL_label }\OperatorTok{==}\StringTok{ "2008"} \OperatorTok{|}
\StringTok{                               }\NormalTok{YBL_label }\OperatorTok{==}\StringTok{ "2009"} \OperatorTok{|}
\StringTok{                               }\NormalTok{YBL_label }\OperatorTok{==}\StringTok{ "2010"} \OperatorTok{|}
\StringTok{                               }\NormalTok{YBL_label }\OperatorTok{==}\StringTok{ "2011"} \OperatorTok{|}
\StringTok{                               }\NormalTok{YBL_label }\OperatorTok{==}\StringTok{ "2012"} \OperatorTok{|}
\StringTok{                               }\NormalTok{YBL_label }\OperatorTok{==}\StringTok{ "2013"} \OperatorTok{|}
\StringTok{                               }\NormalTok{YBL_label }\OperatorTok{==}\StringTok{ "2014"} \OperatorTok{|}
\StringTok{                               }\NormalTok{YBL_label }\OperatorTok{==}\StringTok{ "2015"} \OperatorTok{|}
\StringTok{                               }\NormalTok{YBL_label }\OperatorTok{==}\StringTok{ "2016"} \OperatorTok{|}
\StringTok{                               }\NormalTok{YBL_label }\OperatorTok{==}\StringTok{ "2017"} \OperatorTok{|}
\StringTok{                               }\NormalTok{YBL_label }\OperatorTok{==}\StringTok{ "2018"}\NormalTok{ )) }\OperatorTok
\StringTok{  }
\KeywordTok{mutate}\NormalTok{(}\DataTypeTok{name_of_puma =} \KeywordTok{case_when}\NormalTok{(}
\NormalTok{  PUMA }\OperatorTok{==}\StringTok{ "801"} \OperatorTok{~}\StringTok{ "Far Northeast Heights"}\NormalTok{,}
\NormalTok{  PUMA }\OperatorTok{==}\StringTok{ "802"} \OperatorTok{~}\StringTok{ "Near Northeast Heights"}\NormalTok{, }
\NormalTok{  PUMA }\OperatorTok{==}\StringTok{ "803"} \OperatorTok{~}\StringTok{ "Southeast Heights"}\NormalTok{,}
\NormalTok{  PUMA }\OperatorTok{==}\StringTok{ "804"} \OperatorTok{~}\StringTok{ "Central Abq & North Valley"}\NormalTok{,}
\NormalTok{  PUMA }\OperatorTok{==}\StringTok{ "805"} \OperatorTok{~}\StringTok{ "Northwes Mesa, Paradise Hills, Navajo Nation -Tohajiilee Chapter"}\NormalTok{,}
\NormalTok{  PUMA }\OperatorTok{==}\StringTok{ "806"}\OperatorTok{~}\StringTok{ "Southwest Mesa & South Valley"}\NormalTok{)) }\OperatorTok

\KeywordTok{select}\NormalTok{( struct_old, struct_last_twenty, HINCP ,RAC1P_label, JWMNP, MRGP, RNTP, NP, PUMA, name_of_puma, VALP, AGEP, development, HISP)}
\end{Highlighting}
\end{Shaded}

\hypertarget{ggplot-1}{%
\subsection{GGPlot 1}\label{ggplot-1}}

\begin{Shaded}
\begin{Highlighting}[]
\KeywordTok{options}\NormalTok{(}\DataTypeTok{scipen=}\DecValTok{999}\NormalTok{)}
\KeywordTok{ggplot}\NormalTok{(ass1_vis_vars, }\KeywordTok{aes}\NormalTok{(}\DataTypeTok{x=}\NormalTok{ VALP, }\DataTypeTok{y=}\NormalTok{HINCP, }\DataTypeTok{color =}\NormalTok{ RAC1P_label, }\DataTypeTok{fill=}\NormalTok{RAC1P_label )) }\OperatorTok{+}
\KeywordTok{geom_point}\NormalTok{(}\DataTypeTok{alpha=}\NormalTok{.}\DecValTok{5}\NormalTok{) }\OperatorTok{+}
\StringTok{  }\KeywordTok{labs}\NormalTok{(}\DataTypeTok{title=}\StringTok{"Income and Propery Values"}\NormalTok{, }
         \DataTypeTok{subtitle=}\StringTok{"How doed PV to Income differ along racial demographics"}\NormalTok{,}
         \DataTypeTok{caption=}\StringTok{"Source: ACS1 2018"}\NormalTok{,}
         \DataTypeTok{x=}\StringTok{"Household Income"}\NormalTok{,}
         \DataTypeTok{fill=}\StringTok{"Racee"}\NormalTok{)  }\OperatorTok{+}
\StringTok{  }\KeywordTok{scale_y_continuous}\NormalTok{(}\DataTypeTok{name=} \StringTok{"Income (Thousands)"}\NormalTok{,}
                     \DataTypeTok{breaks =} \KeywordTok{seq}\NormalTok{(}\DecValTok{0}\NormalTok{,}\DecValTok{400000}\NormalTok{, }\DataTypeTok{by =} \DecValTok{50000}\NormalTok{),}
                     \DataTypeTok{labels =} \KeywordTok{paste}\NormalTok{(}\KeywordTok{seq}\NormalTok{(}\DecValTok{0}\NormalTok{,}\DecValTok{400}\NormalTok{, }\DataTypeTok{by =} \DecValTok{50}\NormalTok{),}
                     \StringTok{""}\NormalTok{, }\DataTypeTok{sep =} \StringTok{""}\NormalTok{)) }\OperatorTok{+}
\StringTok{  }\KeywordTok{scale_x_continuous}\NormalTok{(}\DataTypeTok{name =}\StringTok{"Property Values"}\NormalTok{,}
                     \DataTypeTok{breaks =} \KeywordTok{seq}\NormalTok{(}\DecValTok{20000}\NormalTok{, }\DecValTok{900000}\NormalTok{, }\DataTypeTok{by=} \DecValTok{50000}\NormalTok{),}
                     \DataTypeTok{labels =} \KeywordTok{paste}\NormalTok{(}\KeywordTok{seq}\NormalTok{(}\DecValTok{20}\NormalTok{, }\DecValTok{900}\NormalTok{, }\DataTypeTok{by=} \DecValTok{50}\NormalTok{),}
                                    \StringTok{"k"}\NormalTok{, }\DataTypeTok{sep =}\StringTok{""}\NormalTok{)) }\OperatorTok{+}
\StringTok{  }\KeywordTok{scale_color_discrete}\NormalTok{( }\DataTypeTok{name =} \StringTok{"Race"}\NormalTok{) }\OperatorTok{+}\StringTok{ }
\StringTok{  }\KeywordTok{scale_fill_discrete}\NormalTok{(}\DataTypeTok{name =}\StringTok{"Race"}\NormalTok{) }\OperatorTok{+}
\StringTok{  }\KeywordTok{stat_smooth}\NormalTok{(}\DataTypeTok{method =} \StringTok{"lm"}\NormalTok{) }\OperatorTok{+}
\StringTok{  }\KeywordTok{theme_tufte}\NormalTok{()}\OperatorTok{+}
\StringTok{  }\KeywordTok{theme}\NormalTok{(}\DataTypeTok{axis.text.x =} \KeywordTok{element_text}\NormalTok{(}\DataTypeTok{angle =} \DecValTok{90}\NormalTok{))}
\end{Highlighting}
\end{Shaded}

\includegraphics{Graphs_Assignment1_Vis_files/figure-latex/unnamed-chunk-3-1.pdf}

\textbf{Why would I make this graph:} There is an interesting phenomenon
appearing in this chart that income is much more directly correlated to
property values for Asian and White homeowners than it is for Black or
Mixed Race individuals. After calculating this, I wanted to know how
representative my dataset was by getting a better sense of the
demographic breakdown in the area. Which led me to the next chart.

\hypertarget{ggplot-2}{%
\subsection{GGPLOT 2}\label{ggplot-2}}

\begin{Shaded}
\begin{Highlighting}[]
\KeywordTok{ggplot}\NormalTok{(ass1_vis_vars, }\KeywordTok{aes}\NormalTok{(NP, RAC1P_label)) }\OperatorTok{+}
\StringTok{       }\KeywordTok{geom_point}\NormalTok{(}\DataTypeTok{position =} \StringTok{"jitter"}\NormalTok{, }\DataTypeTok{size =}\NormalTok{.}\DecValTok{25}\NormalTok{, }\DataTypeTok{alpha =}\NormalTok{.}\DecValTok{5}\NormalTok{) }\OperatorTok{+}\StringTok{ }
\StringTok{      }\KeywordTok{scale_y_discrete}\NormalTok{(}\DataTypeTok{name =} \StringTok{""}\NormalTok{) }\OperatorTok{+}
\StringTok{      }\KeywordTok{scale_x_continuous}\NormalTok{(}\DataTypeTok{name =} \StringTok{"Number of people in Household, Southeast Heights"}\NormalTok{,}
                                 \DataTypeTok{breaks=} \KeywordTok{seq}\NormalTok{(}\DecValTok{1}\NormalTok{,}\DecValTok{9}\NormalTok{, }\DataTypeTok{by =} \DecValTok{1}\NormalTok{),}
                                \DataTypeTok{labels =} \KeywordTok{paste}\NormalTok{(}\KeywordTok{seq}\NormalTok{(}\DecValTok{1}\NormalTok{,}\DecValTok{9}\NormalTok{, }\DataTypeTok{by =} \DecValTok{1}\NormalTok{), }\StringTok{""}\NormalTok{, }\DataTypeTok{sep=} \StringTok{""}\NormalTok{)) }\OperatorTok{+}
\StringTok{  }\KeywordTok{theme_fivethirtyeight}\NormalTok{()}
\end{Highlighting}
\end{Shaded}

\includegraphics{Graphs_Assignment1_Vis_files/figure-latex/unnamed-chunk-4-1.pdf}

\textbf{Why would I make this graph:} Gives a good sense of both the
racial breakdown in the area, as well as indicating some information
about common household structure.

\hypertarget{gplot-3}{%
\subsection{GPLOT 3}\label{gplot-3}}

\begin{Shaded}
\begin{Highlighting}[]
\KeywordTok{ggplot}\NormalTok{(ass1_vis_vars,}\KeywordTok{aes}\NormalTok{(}\DataTypeTok{x=}\NormalTok{AGEP, }\DataTypeTok{y=}\NormalTok{JWMNP)) }\OperatorTok{+}
\StringTok{  }\KeywordTok{geom_point}\NormalTok{(}\DataTypeTok{size=}\DecValTok{3}\NormalTok{,}\DataTypeTok{alpha=}\NormalTok{.}\DecValTok{7}\NormalTok{) }\OperatorTok{+}\StringTok{ }
\StringTok{  }\KeywordTok{geom_segment}\NormalTok{(}\KeywordTok{aes}\NormalTok{(}\DataTypeTok{x=}\NormalTok{AGEP, }
                   \DataTypeTok{xend=}\NormalTok{AGEP, }
                   \DataTypeTok{y=}\DecValTok{0}\NormalTok{, }
                   \DataTypeTok{yend=}\NormalTok{JWMNP)) }\OperatorTok{+}\StringTok{ }
\StringTok{  }\KeywordTok{labs}\NormalTok{(}\DataTypeTok{title=}\StringTok{"Whats Your Drive?"}\NormalTok{, }
       \DataTypeTok{subtitle=}\StringTok{"Commute Time Versus Age"}\NormalTok{, }
       \DataTypeTok{caption=}\StringTok{"source: ACS1"}\NormalTok{) }\OperatorTok{+}\StringTok{ }
\StringTok{  }\KeywordTok{scale_y_continuous}\NormalTok{(}\DataTypeTok{name=}\StringTok{"Commute Time in Minutes"}\NormalTok{) }\OperatorTok{+}
\StringTok{  }\KeywordTok{scale_x_continuous}\NormalTok{(}\DataTypeTok{name=}\StringTok{"Age of Individual"}\NormalTok{)}
\end{Highlighting}
\end{Shaded}

\includegraphics{Graphs_Assignment1_Vis_files/figure-latex/unnamed-chunk-5-1.pdf}

\begin{Shaded}
\begin{Highlighting}[]
  \KeywordTok{theme}\NormalTok{(}\DataTypeTok{axis.text.x =} \KeywordTok{element_text}\NormalTok{(}\DataTypeTok{angle=}\DecValTok{65}\NormalTok{, }\DataTypeTok{vjust=}\FloatTok{0.6}\NormalTok{)) }\OperatorTok{+}
\StringTok{    }\KeywordTok{theme_fivethirtyeight}\NormalTok{()}
\end{Highlighting}
\end{Shaded}

This graph made me think of the ``productive window'' of time a person
spends working. Very few data points past 75. Also people frequently
seem to round their commute time. Makes me think that I would want to
make this into a categorical variable.

\hypertarget{ggplot-4}{%
\subsection{GGPLOT 4}\label{ggplot-4}}

\begin{Shaded}
\begin{Highlighting}[]
\KeywordTok{ggplot}\NormalTok{(ass1_vis_vars,}\KeywordTok{aes}\NormalTok{(}\DataTypeTok{x=}\NormalTok{RAC1P_label, }\DataTypeTok{fill=}\NormalTok{struct_last_twenty )) }\OperatorTok{+}
\StringTok{  }\KeywordTok{geom_bar}\NormalTok{() }\OperatorTok{+}
\StringTok{  }\KeywordTok{scale_x_discrete}\NormalTok{(}\DataTypeTok{name=}\StringTok{""}\NormalTok{) }\OperatorTok{+}
\StringTok{  }\KeywordTok{scale_fill_discrete}\NormalTok{(}\DataTypeTok{name=}\StringTok{"House Built in the Last Twenty Years"}\NormalTok{)}\OperatorTok{+}
\StringTok{  }\KeywordTok{theme}\NormalTok{(}\DataTypeTok{axis.text.x =} \KeywordTok{element_text}\NormalTok{(}\DataTypeTok{angle =} \DecValTok{90}\NormalTok{))}
\end{Highlighting}
\end{Shaded}

\includegraphics{Graphs_Assignment1_Vis_files/figure-latex/unnamed-chunk-6-1.pdf}
Not sure if this data is hiding information due to the skew of white
alone.

\hypertarget{ggplot-5}{%
\subsection{GGPlot 5}\label{ggplot-5}}

\begin{Shaded}
\begin{Highlighting}[]
\NormalTok{g <-}\StringTok{ }\KeywordTok{ggplot}\NormalTok{(ass1_vis_vars, }\KeywordTok{aes}\NormalTok{(HINCP)) }
\NormalTok{g }\OperatorTok{+}\StringTok{ }\KeywordTok{geom_density}\NormalTok{(}\KeywordTok{aes}\NormalTok{(}\DataTypeTok{fill=}\KeywordTok{factor}\NormalTok{(development)), }\DataTypeTok{alpha=}\FloatTok{0.8}\NormalTok{) }\OperatorTok{+}\StringTok{ }
\StringTok{    }\KeywordTok{labs}\NormalTok{(}\DataTypeTok{title=}\StringTok{"Race and Home Development"}\NormalTok{, }
         \DataTypeTok{subtitle=}\StringTok{"Decade Home was Built and income"}\NormalTok{,}
         \DataTypeTok{caption=}\StringTok{"Source: ACS1 2018"}\NormalTok{,}
         \DataTypeTok{x=}\StringTok{"Household Income"}\NormalTok{,}
         \DataTypeTok{fill=}\StringTok{"Construction Decade"}\NormalTok{) }\OperatorTok{+}
\StringTok{  }\KeywordTok{scale_x_continuous}\NormalTok{(}\DataTypeTok{breaks =} \KeywordTok{seq}\NormalTok{(}\DecValTok{0}\NormalTok{,}\DecValTok{400000}\NormalTok{, }\DataTypeTok{by =} \DecValTok{50000}\NormalTok{),}
                     \DataTypeTok{labels =} \KeywordTok{paste}\NormalTok{(}\KeywordTok{seq}\NormalTok{(}\DecValTok{0}\NormalTok{,}\DecValTok{400}\NormalTok{, }\DataTypeTok{by =} \DecValTok{50}\NormalTok{),}
                     \StringTok{"k"}\NormalTok{, }\DataTypeTok{sep =} \StringTok{""}\NormalTok{)) }
\end{Highlighting}
\end{Shaded}

\includegraphics{Graphs_Assignment1_Vis_files/figure-latex/unnamed-chunk-7-1.pdf}

\begin{Shaded}
\begin{Highlighting}[]
  \KeywordTok{theme}\NormalTok{(}\DataTypeTok{axis.text.x =} \KeywordTok{element_text}\NormalTok{(}\DataTypeTok{angle =} \DecValTok{90}\NormalTok{)) }\OperatorTok{+}
\StringTok{  }\KeywordTok{theme_classic}\NormalTok{()}
\end{Highlighting}
\end{Shaded}

Im still trying to figure out what this one means, but I feel that it
could be useful someday.

\hypertarget{ggplot-6}{%
\subsection{GGplot 6}\label{ggplot-6}}

\begin{Shaded}
\begin{Highlighting}[]
\KeywordTok{ggplot}\NormalTok{(ass1_vis_vars, }\KeywordTok{aes}\NormalTok{(}\DataTypeTok{x=}\NormalTok{HINCP, }\DataTypeTok{y=}\NormalTok{development)) }\OperatorTok{+}
\StringTok{  }\KeywordTok{geom_point}\NormalTok{(}\DataTypeTok{size=}\DecValTok{5}\NormalTok{, }\DataTypeTok{alpha=}\NormalTok{.}\DecValTok{25}\NormalTok{) }\OperatorTok{+}
\StringTok{  }\KeywordTok{scale_x_continuous}\NormalTok{(}\StringTok{"Household Income"}\NormalTok{,}
                      \DataTypeTok{breaks =} \KeywordTok{seq}\NormalTok{(}\DecValTok{0}\NormalTok{,}\DecValTok{400000}\NormalTok{, }\DataTypeTok{by =} \DecValTok{50000}\NormalTok{),}
                     \DataTypeTok{labels =} \KeywordTok{paste}\NormalTok{(}\KeywordTok{seq}\NormalTok{(}\DecValTok{0}\NormalTok{,}\DecValTok{400}\NormalTok{, }\DataTypeTok{by =} \DecValTok{50}\NormalTok{),}
                     \StringTok{"k"}\NormalTok{, }\DataTypeTok{sep =} \StringTok{""}\NormalTok{)) }\OperatorTok{+}\StringTok{ }
\StringTok{  }\KeywordTok{scale_y_discrete}\NormalTok{(}\DataTypeTok{name=}\StringTok{"Decade Home was Constructed"}\NormalTok{) }\OperatorTok{+}\StringTok{ }
\StringTok{  }\KeywordTok{theme_classic}\NormalTok{()}
\end{Highlighting}
\end{Shaded}

\includegraphics{Graphs_Assignment1_Vis_files/figure-latex/unnamed-chunk-8-1.pdf}
SO I thought this map was interesting, in that many of the highest
income earners were in 1950s houses. My next thought was thinking about
my first graph that revealed different correlations to income/property
values along race. So for my next graph, I wanted to bring that variable
in as well.

\hypertarget{ggplot-7}{%
\subsection{GGPLOT 7}\label{ggplot-7}}

\begin{Shaded}
\begin{Highlighting}[]
\KeywordTok{ggplot}\NormalTok{(ass1_vis_vars, }\KeywordTok{aes}\NormalTok{(}\DataTypeTok{x=}\NormalTok{HINCP, }\DataTypeTok{y=}\NormalTok{development, }\DataTypeTok{color=}\NormalTok{RAC1P_label)) }\OperatorTok{+}
\StringTok{  }\KeywordTok{geom_point}\NormalTok{(}\DataTypeTok{size=}\DecValTok{5}\NormalTok{, }\DataTypeTok{alpha=}\NormalTok{.}\DecValTok{25}\NormalTok{) }\OperatorTok{+}
\StringTok{  }\KeywordTok{scale_x_continuous}\NormalTok{(}\StringTok{"Household Income"}\NormalTok{,}
                      \DataTypeTok{breaks =} \KeywordTok{seq}\NormalTok{(}\DecValTok{0}\NormalTok{,}\DecValTok{400000}\NormalTok{, }\DataTypeTok{by =} \DecValTok{50000}\NormalTok{),}
                     \DataTypeTok{labels =} \KeywordTok{paste}\NormalTok{(}\KeywordTok{seq}\NormalTok{(}\DecValTok{0}\NormalTok{,}\DecValTok{400}\NormalTok{, }\DataTypeTok{by =} \DecValTok{50}\NormalTok{),}
                     \StringTok{"k"}\NormalTok{, }\DataTypeTok{sep =} \StringTok{""}\NormalTok{)) }\OperatorTok{+}\StringTok{ }
\StringTok{  }\KeywordTok{scale_y_discrete}\NormalTok{(}\DataTypeTok{name=}\StringTok{"Decade Home was Constructed"}\NormalTok{) }\OperatorTok{+}\StringTok{ }
\StringTok{  }\KeywordTok{theme_classic}\NormalTok{()}
\end{Highlighting}
\end{Shaded}

\includegraphics{Graphs_Assignment1_Vis_files/figure-latex/unnamed-chunk-9-1.pdf}
Interesting! Not entirely sure it is worth the lack of visual clarity,
if developing further, I would categorize into white/non-white.

\hypertarget{ggplot-8}{%
\subsection{GGPlot 8}\label{ggplot-8}}

\begin{Shaded}
\begin{Highlighting}[]
\NormalTok{g <-}\StringTok{ }\KeywordTok{ggplot}\NormalTok{(ass1_vis_vars, }\KeywordTok{aes}\NormalTok{(}\DataTypeTok{x=}\NormalTok{development, }\DataTypeTok{y=}\NormalTok{NP))}
\NormalTok{g }\OperatorTok{+}\StringTok{ }\KeywordTok{geom_violin}\NormalTok{() }\OperatorTok{+}\StringTok{ }
\StringTok{  }\KeywordTok{labs}\NormalTok{(}\DataTypeTok{title=}\StringTok{"House: Deccade Built and Number of People In It"}\NormalTok{, }
       \DataTypeTok{subtitle=}\StringTok{"Were the 80s for large families?"}\NormalTok{,}
       \DataTypeTok{caption=}\StringTok{"Source: ACS1 2018"}\NormalTok{,}
       \DataTypeTok{x=}\StringTok{"Decade Home Constructed"}\NormalTok{,}
       \DataTypeTok{y=}\StringTok{"Number of People In Household"}\NormalTok{) }\OperatorTok{+}
\KeywordTok{theme_fivethirtyeight}\NormalTok{()}
\end{Highlighting}
\end{Shaded}

\includegraphics{Graphs_Assignment1_Vis_files/figure-latex/unnamed-chunk-10-1.pdf}
After making this, I was really excited to see if this would be a BETTER
way of expressing the data in plots 6 \& 7.

\hypertarget{ggplot-9}{%
\subsection{GGPlot 9}\label{ggplot-9}}

\begin{Shaded}
\begin{Highlighting}[]
\NormalTok{g <-}\StringTok{ }\KeywordTok{ggplot}\NormalTok{(ass1_vis_vars, }\KeywordTok{aes}\NormalTok{(}\DataTypeTok{x=}\NormalTok{development, }\DataTypeTok{y=}\NormalTok{HINCP))}
\NormalTok{g }\OperatorTok{+}\StringTok{ }\KeywordTok{geom_violin}\NormalTok{(}\DataTypeTok{siez=}\DecValTok{8}\NormalTok{) }\OperatorTok{+}\StringTok{ }
\StringTok{  }\KeywordTok{labs}\NormalTok{(}\DataTypeTok{title=}\StringTok{"Decade Constructed and Income of Inhabitants"}\NormalTok{, }
       \DataTypeTok{subtitle=}\StringTok{"Rich? Move into a 1950s home?"}\NormalTok{,}
       \DataTypeTok{caption=}\StringTok{"Source: ACS1 2018"}\NormalTok{,}
       \DataTypeTok{x=}\StringTok{"Decade Home Constructed"}\NormalTok{,}
       \DataTypeTok{y=}\StringTok{"Household Income"}\NormalTok{) }\OperatorTok{+}
\StringTok{  }\KeywordTok{scale_y_continuous}\NormalTok{(}\DataTypeTok{breaks =} \KeywordTok{seq}\NormalTok{(}\DecValTok{0}\NormalTok{,}\DecValTok{400000}\NormalTok{, }\DataTypeTok{by =} \DecValTok{50000}\NormalTok{),}
                     \DataTypeTok{labels =} \KeywordTok{paste}\NormalTok{(}\KeywordTok{seq}\NormalTok{(}\DecValTok{0}\NormalTok{,}\DecValTok{400}\NormalTok{, }\DataTypeTok{by =} \DecValTok{50}\NormalTok{),}
                     \StringTok{"k"}\NormalTok{, }\DataTypeTok{sep =} \StringTok{""}\NormalTok{)) }\OperatorTok{+}
\StringTok{  }\KeywordTok{theme_economist_white}\NormalTok{()}
\end{Highlighting}
\end{Shaded}

\begin{verbatim}
## Warning: Ignoring unknown parameters: siez
\end{verbatim}

\includegraphics{Graphs_Assignment1_Vis_files/figure-latex/unnamed-chunk-11-1.pdf}
DEFINITELY an improvement!

\hypertarget{ggplot-10}{%
\subsection{GGPlot 10}\label{ggplot-10}}

\begin{Shaded}
\begin{Highlighting}[]
\KeywordTok{ggplot}\NormalTok{(ass1_vis_vars, }\KeywordTok{aes}\NormalTok{(}\DataTypeTok{x=}\NormalTok{NP, }\DataTypeTok{y=}\NormalTok{JWMNP)) }\OperatorTok{+}\StringTok{ }
\StringTok{  }\KeywordTok{geom_point}\NormalTok{(}\DataTypeTok{col=}\StringTok{"tomato2"}\NormalTok{, }\DataTypeTok{size=}\DecValTok{3}\NormalTok{, }\DataTypeTok{alpha=}\NormalTok{.}\DecValTok{4}\NormalTok{) }\OperatorTok{+}\StringTok{  }
\StringTok{  }\KeywordTok{geom_segment}\NormalTok{(}\KeywordTok{aes}\NormalTok{(}\DataTypeTok{x=}\NormalTok{NP, }
                   \DataTypeTok{xend=}\NormalTok{NP, }
                   \DataTypeTok{y=}\KeywordTok{min}\NormalTok{(JWMNP), }
                   \DataTypeTok{yend=}\KeywordTok{max}\NormalTok{(JWMNP)), }
               \DataTypeTok{linetype=}\StringTok{"dashed"}\NormalTok{, }
               \DataTypeTok{size=}\FloatTok{0.1}\NormalTok{) }\OperatorTok{+}\StringTok{ }
\StringTok{  }\KeywordTok{labs}\NormalTok{(}\DataTypeTok{title=}\StringTok{"Do your roomates make you late?"}\NormalTok{, }
       \DataTypeTok{subtitle=}\StringTok{"Housemates v Commute"}\NormalTok{, }
       \DataTypeTok{caption=}\StringTok{"source: ACS1 2018"}\NormalTok{) }\OperatorTok{+}\StringTok{  }
\StringTok{  }\KeywordTok{scale_x_discrete}\NormalTok{(}\DataTypeTok{name=}\StringTok{"Number of Housemates"}\NormalTok{) }\OperatorTok{+}
\StringTok{  }\KeywordTok{scale_y_continuous}\NormalTok{(}\DataTypeTok{name=}\StringTok{"Minutes in Daily Commute"}\NormalTok{) }\OperatorTok{+}
\StringTok{  }\KeywordTok{coord_flip}\NormalTok{()}
\end{Highlighting}
\end{Shaded}

\includegraphics{Graphs_Assignment1_Vis_files/figure-latex/unnamed-chunk-12-1.pdf}

\hypertarget{ggplot-11-you-promised-we-could-have-a-truly-terrible-one}{%
\subsection{\texorpdfstring{GGPlot 11 --\textgreater{} you promised we
could have a \textbf{truly} terrible
one!}{GGPlot 11 --\textgreater{} you promised we could have a truly terrible one!}}\label{ggplot-11-you-promised-we-could-have-a-truly-terrible-one}}

\begin{Shaded}
\begin{Highlighting}[]
\KeywordTok{ggplot}\NormalTok{(ass1_vis_vars, }\KeywordTok{aes}\NormalTok{(}\DataTypeTok{x=}\NormalTok{NP, }\DataTypeTok{y=}\NormalTok{MRGP)) }\OperatorTok{+}
\StringTok{  }\KeywordTok{geom_line}\NormalTok{(}\DataTypeTok{color=}\StringTok{"green"}\NormalTok{)}
\end{Highlighting}
\end{Shaded}

\includegraphics{Graphs_Assignment1_Vis_files/figure-latex/unnamed-chunk-13-1.pdf}

\hypertarget{r-cheat-sheet-references-troubleshooting}{%
\subsection{R Cheat Sheet, References \&
Troubleshooting}\label{r-cheat-sheet-references-troubleshooting}}

Hello! I see this area of my assignment as predominantly a resource for
myself as I learn R and as a resource to R if I ever need to return
after a hiatus (and have forgotten everything I know).

Feel free to look through, but there is no need to grade anything in
this section.

\textbf{Troubleshooting Checklist} 0. Breathe. 1. Is everything
spelled/cApiTAlized correctly? 2. Count your parentheses(), commas, and
``quotes'' 2.1 Check poles \%\textgreater\%, \textbar{} 3. Have you run
the R chunks that are before the chunk with your error. 5. What is the
last thing you changed? Did you make sure you changed it in ALL the
places it needs to be changed? 6. Breathe. Maybe get a glass of water.
7. Type in the ?function() into the Console 8. Do a Google search for
``R Markdown'' + Your Problem 9. Try copying and pasting sections in
another R Chunk --\textgreater{} Isolate the Problem 10. Phone a friend.

\textbf{Common Errors:} -If you are having a function not found error,
check whether or not you have run your libraries. -YAML error, check the
very beginning of your code -

\hypertarget{references}{%
\subsubsection{References}\label{references}}

\textbf{For General R} R is for Data Science
\url{https://r4ds.had.co.nz/index.html}

Data Wrangling Cheat Sheet
\url{https://rstudio.com/wp-content/uploads/2015/02/data-wrangling-cheatsheet.pdf}

*For GGPLOT** Top 50 Graph Types
\url{http://r-statistics.co/Top50-Ggplot2-Visualizations-MasterList-R-Code.html\#Violin\%20Plot}

GGplot Cheat Sheet
\url{https://rstudio.com/wp-content/uploads/2015/03/ggplot2-cheatsheet.pdf}

Color Pallets
\url{https://www.datanovia.com/en/blog/ggplot-colors-best-tricks-you-will-love/\#predefined-ggplot-color-palettes}

Look into Wes Anderson Themes, WSJ, Economist

\#\#Variables (For Reference)

\begin{Shaded}
\begin{Highlighting}[]
\CommentTok{#Person Level}
\NormalTok{person_vars_}\DecValTok{2018}\NormalTok{ <-}\StringTok{ }\NormalTok{pums_variables }\OperatorTok
\StringTok{  }\KeywordTok{distinct}\NormalTok{(year, survey, var_code, var_label, data_type, level) }\OperatorTok
\StringTok{  }\KeywordTok{filter}\NormalTok{(level}\OperatorTok{==}\StringTok{"person"}\NormalTok{, year}\OperatorTok{==}\DecValTok{2018}\NormalTok{, survey}\OperatorTok{==}\StringTok{"acs1"}\NormalTok{)}

\CommentTok{#HouseHold Level}
\NormalTok{hh_vars_}\DecValTok{2018}\NormalTok{ <-}\StringTok{ }\NormalTok{pums_variables }\OperatorTok
\StringTok{  }\KeywordTok{distinct}\NormalTok{(year, survey, var_code, var_label, data_type, level)}\OperatorTok
\StringTok{  }\KeywordTok{filter}\NormalTok{(level}\OperatorTok{==}\StringTok{"housing"}\NormalTok{, year}\OperatorTok{==}\DecValTok{2018}\NormalTok{, survey}\OperatorTok{==}\StringTok{"acs1"}\NormalTok{)}
\end{Highlighting}
\end{Shaded}

\end{document}
